\documentclass[letterpaper,10pt]{article}
\usepackage{graphicx}
\usepackage{listings}
\usepackage{fullpage}
\usepackage{fixltx2e}
\usepackage{multirow}
\usepackage{amssymb,amsmath}
\usepackage{mathtools}
\usepackage{bm}
\usepackage[hyperfootnotes=false]{hyperref}
\usepackage{url}
\usepackage{subfig}
\usepackage{relsize}
\usepackage{enumitem}
\usepackage{fancyhdr}
\usepackage{framed}
\setlength{\headheight}{14pt}
\pagestyle{fancy}
\headsep = 20pt

% Lineskip mods
\linespread{1.0}
\setlength{\parskip}{0.5\baselineskip}
\setlength{\parindent}{0pt}
\newlength\docparskip
\parskip=6pt
\setlength{\docparskip}{\parskip}
\renewcommand{\arraystretch}{1.085}
\usepackage{xcolor}
\lstset{basicstyle=\ttfamily,
  showstringspaces=false,
  commentstyle=\color{red},
  keywordstyle=\color{blue}
}
\begin{document}

\fancyhf{}
\fancyhead[L]{AME 60614: Numerical Methods}
\fancyhead[R]{Qihao Zhuo: Problem Set 0}
\fancyfoot[C]{\thepage}

\thispagestyle{plain}
\begin{center}
  \large
  \textbf{AME 60614: Numerical Methods} \\
  \textbf{Fall 2021} \\
  \vspace{0.5em}
  \textbf{Problem Set 2} \\
  \vspace{1em}
  Qihao Zhuo
\end{center}

\vspace{1.5em}

All codes are submitted to Sakai, and only their file names will be mentioned in this report when they occurs. 
\section{Finite-Difference Schemes}
\subsection{a}
Taking the determinant of $T_{\alpha}-\lambda I$ with $n \times n$ as $D_n$. Computing $D_n$ by expanding the last row, 
\begin{align}
  D_n &= \left(\alpha-\lambda\right)D_{n-1} - \left(-1\right)^{2n-1}\left(-1\right)^{2n-2}D_{n-2}\notag\\
  D_n &= \left(\alpha-\lambda\right)D_{n-1} - D_{n-2}\notag\\
  D_0 &= 0\notag\\
  D_1 &= \left(\alpha-\lambda\right)\notag
\end{align}

Taking $\left(\alpha-\lambda\right)$ as $2x$, then the recurrence relation of $D_n$ becomes, 
\begin{align}
  D_0 &= 0\notag\\
  D_1 &= 2x\notag\\
  D_n &= 2xD_{n-1} - D_{n-2}\notag
\end{align}

That is the famous Chebyshev polynomials of the second kind, and the roots are, 
\begin{equation}
  x_j = \cos \left(\frac{\pi j}{n+1}\right),~j=1,...,n\notag
\end{equation}

Therefore, the eigenvalues can be solved, 
\begin{align}
  \alpha - \lambda_j &= 2 x_j = 2\cos \left(\frac{\pi j}{n+1}\right)\notag\\
  \lambda_j &= \alpha - 2 \cos \left(\frac{\pi j}{n+1}\right),~j=1,...,n\notag
\end{align}

For $k-th$ component of eigenvector associated with $\lambda_j$, 
\begin{equation}
  x_{k+2}-\left(\alpha - \lambda_j\right)x_{k+1}+x_k=0\notag
\end{equation}

Such equation has different eigenroots. Therefore, 
\begin{align}
  x_0 &=x_{n+1} = 0\notag\\
  x_k &= ar_1^k + br_2^k\notag\\
  a+b &=0\notag\\
  x_k &= a\left(e^{\frac{ik\pi j}{n+1}}-e^{-\frac{ik\pi j}{n+1}}\right) = 2ia \sin \left(\frac{k\pi j}{n+1}\right)\notag
\end{align}

Since $a$ is arbitrary, setting $a = \frac{1}{2i}$. 
\begin{align}
  x_k &= \sin \left(\frac{k\pi j}{n+1}\right)\notag\\
  \bm{q}_j &= \left[\sin \left(\frac{\pi j}{n+1}\right),\sin \left(\frac{2\pi j}{n+1}\right),...,\sin \left(\frac{n\pi j}{n+1}\right)\right]^T\notag
\end{align}
\subsection*{b}

\end{document}